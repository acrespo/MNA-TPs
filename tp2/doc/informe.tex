\documentclass[twocolumn,a4paper,10pt]{article}

\usepackage[utf8]{inputenc}
\usepackage{t1enc}
\usepackage[spanish]{babel}
\usepackage[pdftex,usenames,dvipsnames]{color}
\usepackage[pdftex]{graphicx}
\usepackage{enumerate}
\usepackage{url}
\usepackage{amsmath}
\usepackage{amsfonts}
\usepackage{amssymb}
\usepackage[table]{xcolor}
\usepackage[small,bf]{caption}
\usepackage{float}
\usepackage{subfig}
\usepackage{bm}
\usepackage{fancyhdr}
\usepackage{times}
\usepackage{titlesec}
\usepackage[numbers]{natbib}
\usepackage{titling}
\usepackage{listings}

\renewcommand{\lstlistingname}{Código Fuente}

%%% Listings
\lstloadlanguages{Octave} 
\lstdefinelanguage{MyOctave}[]{Octave}{% 
	deletekeywords={beta,det},
	morekeywords={repmat}
} 
\lstset{ %
	language=MyOctave,
	stringstyle=\ttfamily,
	showstringspaces = false,
	basicstyle=\footnotesize\ttfamily,
	commentstyle=\color{gray},
	keywordstyle=\bfseries,
	numbers=left,
	numberstyle=\ttfamily\footnotesize,
	stepnumber=1,                   % the step between two line-numbers. If it's 1 each line will be numbered
	framexleftmargin=0.10cm,
	numbersep=0.05cm,               % how far the line-numbers are from the code
	backgroundcolor=\color{white},
	showspaces=false,
	showtabs=false,
% 	frame=l,
	tabsize=4,
	captionpos=b,                   % sets the caption-position to bottom
	breaklines=true,                % sets automatic line breaking
	breakatwhitespace=false,        % sets if automatic breaks should only happen at whitespace
	mathescape=true
}

\def\customabstract{\vspace{.5em}
    {\small\center{\textbf{RESUMEN}} \\[0.5em] \relax%
}}
\def\endkeywords{\par}

\def\keywords{\vspace{.5em}
    {\textit{Palabras clave: } 
}}
\def\endkeywords{\par}

\titleformat{\section}{\small\center\bfseries}{\thesection.}{0.5em}{\normalsize\uppercase}
\titleformat{\subsection}{\small\center\bfseries}{}{0.5em}{\small\uppercase}
\renewcommand{\bibsection}{}

% TITLE Configuration
\setlength{\droptitle}{-30pt}
\pretitle{\begin{center}\Huge\begin{rmfamily}}
\posttitle{\par\end{rmfamily}\end{center}\vskip 0.5em}
\preauthor{\begin{center}
        \large \lineskip 0.5em%
\begin{tabular}[t]{c}}
\postauthor{\end{tabular}\normalsize 
    \\[1em] Estudiantes del Instituto Tecnológico de Buenos Aires
\par\end{center}}
\predate{\begin{center}\small}
\postdate{\par\end{center}}

% Headers
\addtolength{\voffset}{-40pt}
\addtolength{\textheight}{80pt}
\renewcommand{\headrulewidth}{0pt}
\fancyhead{}
\fancyfoot{}
\lhead{\small No publicado: Cátedra de Métodos Numéricos Avanzados (ITBA)}
\rhead{\small \thepage}
\cfoot{\small Copyright \copyright 2012 ITBA}

% Metadata
\title{Procesamiento de Imagenes}
\date{23 de Noviembre de 2012}
\author{Civile, Juan Pablo \and Crespo, Álvaro \and Ordano, Esteban }

\begin{document}

\pagestyle{fancy}
\maketitle
\thispagestyle{fancy}

\begin{customabstract}
\textbf{
El objetivo del presente trabajo es realizar un an\'alisis espectral de una imagen y utilizar la Transformada Discreta de Fourier en 2 dimensiones para 
la aplicaci\'on de filtros en el dominio de frecuencias, también conocidos como espaciales.
}
\end{customabstract}

\begin{keywords}
Transformada Discreta de Fourier en 2 dimensiones, Filtros espaciales, Fast Fourier Transform, an\'alisis espectral
\end{keywords}

\section{Introducci\'on}

Una imagen en escala de grises, como la que se desea estudiar, puede pensarse como una funci\'on en 2 dimensiones $f(x,y)$ donde el valor de la función
para cada $x,y$ está dado por la intensidad luz en esa posici\'on. Dicha intensidad se mide de 0 (ausencia de luz, color negro) a 255 (m\'axima intensidad, 
color blanco). En el presente trabajo, se utilizará esta concepci\'on para la aplicaci\'on de filtros espaciales, via la 
Transformada Discreta de Fourier en 2 dimensiones. \\

La Transformada Discreta de Fourier en 2 dimensiones, así como la Transformada Inversa Discreta, se encuentran definidas en la secci\'on \\
% TODO insert seccion here
En la secci\'on 
% TODO insert seccion here
se explica la metodología usada para la obtenci\'on de las im\'agenes correspondientes a la amplitud y a la fase, y para aplicaci\'on de los filtros. \\
En la secci\'on 
% TODO insert seccion here
se muestran los resultados obtenidos, junto con el an\'alisis de cada uno. \\
Todos los resultados se obtienen a partir de la Figura
% TODO insert figura here
, la cual es una imagen en escala de grises del planeta Saturno. \\

\section{Transformada Discreta de Fourier en 2 dimensiones}
La Trasnformada Discreta de Fourier de una secuencia bidimensional $x_{n,m}$ de $N \times N$ (en nuestro caso, la imagen)

\begin{equation}
    X_{l,k} = \sum_{n=0}^{N - 1} \sum_{m=0}^{N - 1} x_{n,m} e^{-i*\frac{2*\pi}{N}*(nl + mk)}
\end{equation}

siendo la Transformada Inversa Discreta

\begin{equation}
    x_{n,m} = \frac{1}{N^2} \sum_{l=0}^{N - 1} \sum_{k=0}^{N - 1} X_{l,k} e^{i*\frac{2*\pi}{N}*(nl + mk)}
\end{equation}

En la secci\'on 
% TODO insert seccion here
se encuentra nuestra implementaci\'on, tanto de la Trasnformada Discreta como de la Transformada Inversa Discreta. Sin embargo, es f\'acil ver que estas 
implementaciones son de orden $O(n^2)$ para calcular cada elemento de la matriz, por lo que aplicar nuestra implementaci\'on de la 
transformada (o de la transformada inversa) ser\'ia orden $O(n^4)$. Teniendo esto en cuenta, y agregando el hecho de que nuestra imagen es de 400 $x$ 400 
pixeles, se hace necesario recurrir a una implementaci\'on m\'as eficiente. Es por esto que para la obtenci\'on de los resultados y su posterior an\'alisis 
se utilizan las funciones \textit{built-in} de \textit{Octave}, \textit{fft2} y \textit{ifft2}, que implementan algoritmos del estilo FFT 
(\textit{Fast Fourier Transform}). \\

En la secci\'on 
% TODO insert seccion here
se deja para el lector el c\'odigo fuente de un sencillo programa de prueba que compara, para una matriz de un tamaño pequeño, los resultados obtenidos a 
partir de nuestras implementaciones contra los de las funciones de \textit{Octave}.\\

\section{Metodolog\'ia}

\subsection{Imagen en amplitud y en fase. Recupero de imagen original.}

Para hallar la imagen en amplitud, se procede aplicando la Transformada Discreta de Fourier en 2 dimensiones a nuestra imagen y qued\'andonos con el 
m\'odulo. Cabe destacar que la Transformada Discreta devuelve n\'umeros complejos. \\

Para hallar la imagen en fase, el proceso es el mismo, excepto que en lugar de quedarnos con el m\'odulo de los n\'umeros complejos, nos quedamos con el 
\'angulo o argumento. Tambi\'en se agrega una normalizaci\'on, para mapear los valores de fase al intervalo entero $[0, 255]$, 
requerido para representar la escala de grises.\\

Finalmente, se aplica la Transformada Inversa Discreta a la imagen transformada para reconstrui la imagen original.\\

\subsection{Aplicaci\'on de filtros}

Para la aplicaci\'on de un filtro sobre la imagen se requiere que la funci\'on que act\'ua como filtro est\'e bien definida para todo elemento de la imagen.
Teniendo esto en cuenta, se obtiene la imagen transformada (imagen en frecuencias), se multiplica \textit{elemento a elemento} por el filtro deseado, y 
finalmente se aplica la Transformada Inversa Discreta para recuperar la imagen fitrada. \\

A continuaci\'on se explicitan las funciones de los filtros analizados en este documento. \\

\subsubsection{Filtro Custom}

\begin{equation}
    H_{k,l} = \left\{
                    \begin{array}{c l}
                        0 & \mbox{si } 0 \leq k \leq 400, 190 \leq l \leq 210\\
                        0 & \mbox{si } 0 \leq l \leq 400, 190 \leq k \leq 210\\
                        1 & \mbox{en todo otro caso}
                    \end{array}               
               \right.
\end{equation}

\subsubsection{Filtro Gaussiano}

\begin{equation}
    H_{k,l} = exp(-0,1(k^2 + l^2))
\end{equation}


\subsubsection{Filtro Damero}

\begin{equation}
    H_{k,l} = \left\{
                    \begin{array}{c l}
                        0 & \mbox{si } l+k \mbox{ es par} \\
                        1 & \mbox{si } l+k \mbox{ es impar}
                    \end{array}               
               \right.
\end{equation}

\section{Resultados}

\subsection{Imagen en amplitud y en fase. Recupero de imagen original.}

Como bien dice la secci\'on 
% TODO insert seccion here
tras la aplicaci\'on de la Transformada Discreta de Fourier y filtrando por el m\'odulo y la fase se obtienen las im\'agenes correspondientes a la amplitud
(Figura )
% TODO insert figura here
y a la fase (Figura )
% TODO insert figura here
respectivamente. \\

Luego, aplicando la Trasnformada Inversa Discreta a la imagen transformada, se recupera la imagen original (Figura )
% TODO insert figura here

\subsection{Aplicaci\'on de filtros}

Utilizando la metodolog\'ia explicada en la secci\'on
% TODO insert seccion here
se obtienen las imagenes luego de la aplicaci\'on de cada filtro. \\

\subsubsection{Filtro Custom}

La imagen resultante luego de aplicar el fitro Custom de la secci\'on
% TODO insert seccion here
se presenta en la Figura 
% TODO insert figura here


\subsubsection{Filtro Gaussiano}
La imagen resultante luego de aplicar el fitro Gaussiano de la secci\'on
% TODO insert seccion here
se presenta en la Figura 
% TODO insert figura here

\subsubsection{Filtro Damero}

La imagen resultante luego de aplicar el fitro Damero de la secci\'on 
% TODO insert seccion here
se presenta en la Figura 
% TODO insert figura here

\section{Conclusiones (????)}

\section*{Referencias}

\begin{thebibliography}{99}
\end{thebibliography}

\newpage
\section*{Anexo 1: C\'odigo Fuente}

\newpage
\section*{Anexo 2: Imagenes}

\end{document}
