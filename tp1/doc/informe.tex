\documentclass[twocolumn,a4paper,10pt]{article}

\usepackage[utf8]{inputenc}
\usepackage{t1enc}
\usepackage[spanish]{babel}
\usepackage[pdftex,usenames,dvipsnames]{color}
\usepackage[pdftex]{graphicx}
\usepackage{enumerate}
\usepackage{url}
\usepackage{amsmath}
\usepackage{amsfonts}
\usepackage{amssymb}
\usepackage[table]{xcolor}
\usepackage[small,bf]{caption}
\usepackage{float}
\usepackage{subfig}
\usepackage{bm}
\usepackage{fancyhdr}
\usepackage{times}
\usepackage{titlesec}
\usepackage[numbers]{natbib}
\usepackage{titling}
\usepackage{listings}

\renewcommand{\lstlistingname}{Código Fuente}

%%% Listings
\lstloadlanguages{Octave} 
\lstdefinelanguage{MyOctave}[]{Octave}{% 
	deletekeywords={beta,det},
	morekeywords={repmat}
} 
\lstset{ %
	language=MyOctave,
	stringstyle=\ttfamily,
	showstringspaces = false,
	basicstyle=\footnotesize\ttfamily,
	commentstyle=\color{gray},
	keywordstyle=\bfseries,
	numbers=left,
	numberstyle=\ttfamily\footnotesize,
	stepnumber=1,                   % the step between two line-numbers. If it's 1 each line will be numbered
	framexleftmargin=0.20cm,
	numbersep=0.37cm,               % how far the line-numbers are from the code
	backgroundcolor=\color{white},
	showspaces=false,
	showtabs=false,
	frame=l,
	tabsize=4,
	captionpos=b,                   % sets the caption-position to bottom
	breaklines=true,                % sets automatic line breaking
	breakatwhitespace=false,        % sets if automatic breaks should only happen at whitespace
	mathescape=true
}

\def\customabstract{\vspace{.5em}
    {\small\center{\textbf{RESUMEN}} \\[0.5em] \relax%
}}
\def\endkeywords{\par}

\def\keywords{\vspace{.5em}
    {\textit{Palabras clave: } 
}}
\def\endkeywords{\par}

\titleformat{\section}{\small\center\bfseries}{\thesection.}{0.5em}{\normalsize\uppercase}
\renewcommand{\thesection}{\Roman{section}}
\renewcommand{\bibsection}{}

% TITLE Configuration
\setlength{\droptitle}{-30pt}
\pretitle{\begin{center}\Huge\begin{rmfamily}}
\posttitle{\par\end{rmfamily}\end{center}\vskip 0.5em}
\preauthor{\begin{center}
        \large \lineskip 0.5em%
\begin{tabular}[t]{c}}
\postauthor{\end{tabular}\normalsize 
    \\[1em] Estudiantes del Instituto Tecnológico de Buenos Aires
\par\end{center}}
\predate{\begin{center}\small}
\postdate{\par\end{center}}

% Headers
\addtolength{\voffset}{-40pt}
\addtolength{\textheight}{80pt}
\renewcommand{\headrulewidth}{0pt}
\fancyhead{}
\fancyfoot{}
\lhead{\small No publicado: Cátedra de Métodos Numéricos Avanzados (ITBA)}
\rhead{\small \thepage}
\cfoot{\small Copyright \copyright 2012 ITBA}

% Metadata
\title{Autovalores y Compresión de Imágenes}
\date{20 de Septiembre de 2012}
\author{Civile, Juan Pablo \and Crespo, Álvaro \and Ordano, Esteban }

\begin{document}

\pagestyle{fancy}
\maketitle
\thispagestyle{fancy}

\begin{customabstract}
\textbf{Una cita: \cite{PHPBuch} Lorem ipsum dolor sit amet, consectetur adipiscing elit. Pellentesque adipiscing, orci non ultricies malesuada, lacus enim varius odio, a venenatis justo risus eu ante. Sed a nisi quam, ut placerat leo. Pellentesque habitant morbi tristique senectus et netus et malesuada fames ac turpis egestas.}
\end{customabstract}

\begin{keywords}
paper, paper, paper
\end{keywords}


\section{Introducci\'on}

    Nullam quis odio erat. Cum sociis natoque penatibus et magnis dis parturient montes, nascetur ridiculus mus. Aenean eu mauris ac augue volutpat pellentesque. Donec ultricies pulvinar ipsum sit amet pellentesque. Maecenas eget elit tortor. Donec in enim magna. Donec suscipit enim vitae erat dignissim fringilla nec eget metus. Praesent enim elit, gravida vitae feugiat in, vehicula a velit. Nunc non enim enim, eget accumsan nisl. Pellentesque malesuada velit eu tellus interdum dapibus. In ultricies est at neque imperdiet scelerisque. Lorem ipsum dolor sit amet, consectetur adipiscing elit. Aliquam blandit tempor aliquet. Nam id lacus lacus. Quisque venenatis risus id nunc bibendum hendrerit molestie turpis aliquet. Quisque sit amet sem ac leo iaculis tempus.

    Mauris suscipit gravida justo a laoreet. Proin bibendum nunc eu augue tristique et volutpat enim sollicitudin. Sed sed nulla id metus ultricies sagittis ut et diam. Vestibulum nec nisi ac est euismod cursus ac ac augue. Nullam molestie sodales est, vel pretium libero cursus et. Etiam id elit justo, eu consequat erat. Duis vitae neque metus, sit amet suscipit urna. Vivamus scelerisque lorem eros.

\section{Metodología}
% Pueden ser varias secciones
Nulla facilisi. Vestibulum ante ipsum primis in faucibus orci luctus et ultrices posuere cubilia Curae; Nunc nec elit id diam molestie gravida at sit amet ligula. Nulla id urna eget libero imperdiet aliquam. Integer feugiat elementum sem in euismod. Sed porta mollis magna, laoreet tincidunt est laoreet eget. Aenean nunc magna, iaculis eu imperdiet pellentesque, lobortis eu elit. Duis congue rutrum tellus, nec sodales ipsum sagittis ac. Aliquam et nibh vel justo lacinia luctus eu ut velit. Mauris vel justo mi, quis fermentum mi. Nunc sodales tellus quis magna aliquet vel adipiscing augue semper. Nunc facilisis pharetra elit in ornare. Sed quis nunc sagittis turpis fringilla laoreet at a magna.
\begin{table}
    \center
    \begin{tabular}{c|c|c}
            $t$ [secs] & $\beta$ & $\alpha_1$ \\ \hline \hline
       0.0000000  &  0.000000  & -0.4510268 \\
       0.0785398  &  0.785398  & -0.1532732 \\
       0.1570796  &  1.570796  &  0.0131000 \\
       0.2356194  &  2.356194  &  0.2125534 \\
       0.3141592  &  3.141592  &  1.3363289 \\
       0.3926990  &  3.926990  &  3.5220433 \\
       0.4712388  &  4.712388  &  4.4038604 \\
       0.5497787  &  5.497787  &  5.1487891 \\

    \end{tabular}
    \caption{Valores obtenidos para algunos instantes}
    \label{tabla_segundo_punto}
\end{table}

\subsection{Compresión bruta 2: PCA}

Desde el punto de vista del  \'algebra lineal, la imagen dividida en bloques de 16 x 16, como en el caso anterior, se puede considerar simplemente como
1024 vectores de 256 elementos cada uno (la matriz b del código). Es decir, tenemos vectores en $\mathbb{R}^{256}$ . Dada una base cualquiera de $\mathbb{R}^{256}$,
 que cuenta con 256 vectores linealmente independientes, podemos representar cada uno de los bloques de la imagen por su proyecci\'on sobre cada vector en la base. 
Es decir, si queremos darle la imagen a alguien, le podemos dar la base y los 256 n\'umeros correspondientes a la proyeccci\'on sobre cada vector de la misma. \\

Para reducir el problema, podríamos pensar en usar $N < 256$ vectores linealmente independientes en la base. De esta manera, para comunicar la imagen s\'olo
tenemos que dar N vectores en la base y N n\'umeros por cada bloque. El problema es que usando s\'olo N vectores no se puede representar todo $\mathbb{R}^{256}$ y la 
imagen comunicada no ser\'ia exactamente igual a la original. Se plantea entonces el interrogante de como elegir los N vectores de manera de minimizar la p\'erdida 
de calidad de la imagen. La respuesta la da lo que se denomina \textit{Principal Component Analysis} y, en nuestro caso, es m\'as o menos como sigue.\\

Introducimos el famoso concepto de la Estadística, la Covarianza. La covarianza nos da una idea de qu\'e tanto cambian juntas dos variables. Si tenemos, por ej., 
dos variables aleatorias Y, Z la covarianza se define como,

\[ cov(Y, Z) = E [(Y - E(Y )) (Z - E(Z))] \]

donde $E(x)$ es la esperanza matem\'atica o valor esperado. \\

En nuestro caso, por ej., podemos considerar a cada uno de los 256 n\'umeros que representan el nivel de gris de un pixel en un bloque como un variable aleatoria.
Llamemos $X_{i}, i = 1, 2, \dotsc, 256$, a dichas variables aleatorias. Luego, podemos definir una matriz $C = (C_{ij} )$ de la siguiente manera,

\[C_{ij} = Cov(X_{i}, X_{j})  \]

A $\textbf{C}$ se la denomina matriz de covarianza.\\

Sin embargo, nosotros no tenemos distribuciones de probabilidad que nos permitan calcular la matriz covarianza. S\'olo tenemos, en el ejemplo de la imagen
de Lena, 1024 muestras de cada variable aleatoria: una por cada bloque. Esto nos permite calcular la matriz de covarianza muestral $\widehat{C} = (\widehat{C}_{ij})$:

\[ \widehat{C}_{ij} =  \frac{1}{1024 - 1}\sum_{k=1}^{1024} (x_{ik} - \widehat{x}_{i}) (x_{jk} - \widehat{x}_{j}) \]

donde
\begin{itemize}
    \item $x_{ik}$ es la k-\'esima muestra de la i- \'esima variable aleatoria o, en el contexto de nuestra imagen, el nivel de gris del i-\'esimo pixel en el 
    k- \'esimo bloque,
    \item $\widehat{x}_{i}$ es la media muestral de la i- \'esima variable aleatoria o el nivel de gris del i-\'esimo pixel del “bloque promedio”, es decir,
    \[ \widehat{x}_{i} = \frac{1}{1024} \sum_{k=1}^{1024} x_{ik} \]
\end{itemize}

\paragraph*{}

La matriz de covarianza muestral 
$\widehat{C}_{ij}$ nos habla de qu\'e tanto var\'ian juntos los niveles de gris del i-\'esimo y el j-\'esimo pixel en los bloques de la imagen. Y ahora viene 
el punto clave: vamos a calcular los autovalores y autovectores. Cada autovector nos da una direcci\'on caracter\'istica de cambio y el autovalor correspondiente 
(siempre $\geq$ 0) \footnote{}
 nos dice qu\'e tan importante es el cambio en dicha direcci\'on. En el contexto de la imagen de Lena, cada autovector es una \textit{autoimagen}
que nos habla de un esquema de cambio caracter\'istico de los niveles de gris de los bloques y cada autovalor nos habla de la importancia de ese esquema frente 
a los dem\'as. \\

Vamos a unir todo esto con la compresi\'on de la imagen.Los autovectores conforman la base de $\mathbb{R}^{256}$ de la que habl\'abamos m\'as arriba. 
Los autovalores nos dicen c\'omo elegir un subconjunto de esos autovectores: elegimos aquellos que tienen el mayor autovalor. \\

Nuestro c\'odigo de Octave \ref{codigo-compresionBruta2} muestra c\'omo realizar la compresi\'on usando estas ideas. \\

Podemos ver como se usa la función $cov$ para calcular la matrix de covarianza muestral $\widehat{C}$. A la que luego se le calculan los autovalores y autovectores
utilizando una funci\'on recibida como par\'ametro del programa, $eig\_function$, que simplemente puede ser la funci\'on $eig$ de Octave o una definida por el usuario.
Utilizando la función $diag$ se extraen en un vector los autovalores de la diagonal de la matriz $D$ retornada por $eig\_function$. Dicho vector es ordenado de 
mayor a menor (de forma \textit{descendente}) usando la funci\'on $sort$ y luego se vuelve a utilizar la funci\'on $diag$ para obtener una matriz diagonal, cuya 
diagonal principal sea el vector ordenado. Una vez ordenados los autovalores, resta ordenar de la misma forma los autovectores. Esto se consiguen haciendo $V = V(:,i)$,
donde $i$ es el vector de los \'indices originales que es una de las cosas que devuelve la funci\'on $sort$ al ordenar a los autovalores.\\

Habiendo hecho esto, se procede a obtener las N proyecciones de los N autovectores correspondientes a los N autovalores más grandes. Esto se logra a través de la 
l\'inea $pv(:,i) = V(:,i).' * ds;$. \\

Luego, se le suman estas proyecciones a los valores de la matriz $M$ del caso anterior (y llam\'andola ahora $d$), la cual contiene el ``gris promedio'' para todos 
los vectores. Esto se logra con un bloque $for$ de 1 a 1024, la cantidad de bloques en que dividimos la imagen, que se corresponde con la cantidad de columnas 
de la matriz $d$. En cada iteración de dicho bloque $for$, se deben sumar las N proyeccciones deseadas. Para ello, se necesita otro bloque $for$ de 1 a N, la cantidad 
de proyecciones. La suma de las proyecciones se consigue f\'acilmente mediante $d(:,k)\: += pv(k,i) * V(:,i);$. \\

Finalmente, s\'olo queda volver a juntar las columnas de la matriz $d$ en bloques no solapados para volver a formar la imagen. Esto se logra usando la funci\'on 
$col2im$ de Octave, a la cual se le pasa como par\'ametro el tamaño de bloque, el de la matriz y un flag para indicar que se desean bloques no solapados. Como toque 
final, los n\'umeros de la matriz resultante, la imagen comprimida, son redondeados y convertidos a enteros de 8 bits (para que queden en el rango 0 - 255).\\

Fusce suscipit suscipit erat, et ultricies purus tincidunt sit amet. Sed ut mauris id nisl convallis mattis sit amet ut ante. Vivamus posuere faucibus urna vitae volutpat. Aliquam erat volutpat. Etiam molestie neque vel orci sodales in tincidunt ligula viverra. Aliquam erat volutpat. Morbi sit amet arcu dui. Vivamus eu tellus felis.

\begin{figure}
    \center \includegraphics[height=100pt]{bpositivo.png}
    \caption{Esta figura no muestra nada}
\end{figure}

\section{Resultados}
% Podria ser Resultados y Conclusiones
Nulla facilisi. Vestibulum ante ipsum primis in faucibus orci luctus et ultrices posuere cubilia Curae; Nunc nec elit id diam molestie gravida at sit amet ligula. Nulla id urna eget libero imperdiet aliquam. Integer feugiat elementum sem in euismod. Sed porta mollis magna, laoreet tincidunt est laoreet eget. Aenean nunc magna, iaculis eu imperdiet pellentesque, lobortis eu elit. Duis congue rutrum tellus, nec sodales ipsum sagittis ac. Aliquam et nibh vel justo lacinia luctus eu ut velit. Mauris vel justo mi, quis fermentum mi. Nunc sodales tellus quis magna aliquet vel adipiscing augue semper. Nunc facilisis pharetra elit in ornare. Sed quis nunc sagittis turpis fringilla laoreet at a magna.

Fusce suscipit suscipit erat, et ultricies purus tincidunt sit amet. Sed ut mauris id nisl convallis mattis sit amet ut ante. Vivamus posuere faucibus urna vitae volutpat. Aliquam erat volutpat. Etiam molestie neque vel orci sodales in tincidunt ligula viverra. Aliquam erat volutpat. Morbi sit amet arcu dui. Vivamus eu tellus felis.
% Podria ser Resultados y Conclusiones
Nulla facilisi. Vestibulum ante ipsum primis in faucibus orci luctus et ultrices posuere cubilia Curae; Nunc nec elit id diam molestie gravida at sit amet ligula. Nulla id urna eget libero imperdiet aliquam. Integer feugiat elementum sem in euismod. Sed porta mollis magna, laoreet tincidunt est laoreet eget. Aenean nunc magna, iaculis eu imperdiet pellentesque, lobortis eu elit. Duis congue rutrum tellus, nec sodales ipsum sagittis ac. Aliquam et nibh vel justo lacinia luctus eu ut velit. Mauris vel justo mi, quis fermentum mi. Nunc sodales tellus quis magna aliquet vel adipiscing augue semper. Nunc facilisis pharetra elit in ornare. Sed quis nunc sagittis turpis fringilla laoreet at a magna.

Fusce suscipit suscipit erat, et ultricies purus tincidunt sit amet. Sed ut mauris id nisl convallis mattis sit amet ut ante. Vivamus posuere faucibus urna vitae volutpat. Aliquam erat volutpat. Etiam molestie neque vel orci sodales in tincidunt ligula viverra. Aliquam erat volutpat. Morbi sit amet arcu dui. Vivamus eu tellus felis.
% Podria ser Resultados y Conclusiones
Nulla facilisi. Vestibulum ante ipsum primis in faucibus orci luctus et ultrices posuere cubilia Curae; Nunc nec elit id diam molestie gravida at sit amet ligula. Nulla id urna eget libero imperdiet aliquam. Integer feugiat elementum sem in euismod. Sed porta mollis magna, laoreet tincidunt est laoreet eget. Aenean nunc magna, iaculis eu imperdiet pellentesque, lobortis eu elit. Duis congue rutrum tellus, nec sodales ipsum sagittis ac. Aliquam et nibh vel justo lacinia luctus eu ut velit. Mauris vel justo mi, quis fermentum mi. Nunc sodales tellus quis magna aliquet vel adipiscing augue semper. Nunc facilisis pharetra elit in ornare. Sed quis nunc sagittis turpis fringilla laoreet at a magna.

Fusce suscipit suscipit erat, et ultricies purus tincidunt sit amet. Sed ut mauris id nisl convallis mattis sit amet ut ante. Vivamus posuere faucibus urna vitae volutpat. Aliquam erat volutpat. Etiam molestie neque vel orci sodales in tincidunt ligula viverra. Aliquam erat volutpat. Morbi sit amet arcu dui. Vivamus eu tellus felis.

\begin{figure}
    \center \includegraphics[height=100pt]{bpositivo.png}
    \caption{Esta figura no muestra nada}
\end{figure}

Nulla facilisi. Vestibulum ante ipsum primis in faucibus orci luctus et ultrices posuere cubilia Curae; Nunc nec elit id diam molestie gravida at sit amet ligula. Nulla id urna eget libero imperdiet aliquam. Integer feugiat elementum sem in euismod. Sed porta mollis magna, laoreet tincidunt est laoreet eget. Aenean nunc magna, iaculis eu imperdiet pellentesque, lobortis eu elit. Duis congue rutrum tellus, nec sodales ipsum sagittis ac. Aliquam et nibh vel justo lacinia luctus eu ut velit. Mauris vel justo mi, quis fermentum mi. Nunc sodales tellus quis magna aliquet vel adipiscing augue semper. Nunc facilisis pharetra elit in ornare. Sed quis nunc sagittis turpis fringilla laoreet at a magna.
\begin{figure}
    \center \includegraphics[height=100pt]{bpositivo.png}
    \caption{Esta figura no muestra nada}
\end{figure}

Fusce suscipit suscipit erat, et ultricies purus tincidunt sit amet. Sed ut mauris id nisl convallis mattis sit amet ut ante. Vivamus posuere faucibus urna vitae volutpat. Aliquam erat volutpat. Etiam molestie neque vel orci sodales in tincidunt ligula viverra. Aliquam erat volutpat. Morbi sit amet arcu dui. Vivamus eu tellus felis.

\section{Conclusiones}
Nulla facilisi. Vestibulum ante ipsum primis in faucibus orci luctus et ultrices posuere cubilia Curae; Nunc nec elit id diam molestie gravida at sit amet ligula. Nulla id urna eget libero imperdiet aliquam. Integer feugiat elementum sem in euismod. Sed porta mollis magna, laoreet tincidunt est laoreet eget. Aenean nunc magna, iaculis eu imperdiet pellentesque, lobortis eu elit. Duis congue rutrum tellus, nec sodales ipsum sagittis ac. Aliquam et nibh vel justo lacinia luctus eu ut velit. Mauris vel justo mi, quis fermentum mi. Nunc sodales tellus quis magna aliquet vel adipiscing augue semper. Nunc facilisis pharetra elit in ornare. Sed quis nunc sagittis turpis fringilla laoreet at a magna.

Fusce suscipit suscipit erat, et ultricies purus tincidunt sit amet. Sed ut mauris id nisl convallis mattis sit amet ut ante. Vivamus posuere faucibus urna vitae volutpat. Aliquam erat volutpat. Etiam molestie neque vel orci sodales in tincidunt ligula viverra. Aliquam erat volutpat. Morbi sit amet arcu dui. Vivamus eu tellus felis.
% Podria ser Resultados y Conclusiones
Nulla facilisi. Vestibulum ante ipsum primis in faucibus orci luctus et ultrices posuere cubilia Curae; Nunc nec elit id diam molestie gravida at sit amet ligula. Nulla id urna eget libero imperdiet aliquam. Integer feugiat elementum sem in euismod. Sed porta mollis magna, laoreet tincidunt est laoreet eget. Aenean nunc magna, iaculis eu imperdiet pellentesque, lobortis eu elit. Duis congue rutrum tellus, nec sodales ipsum sagittis ac. Aliquam et nibh vel justo lacinia luctus eu ut velit. Mauris vel justo mi, quis fermentum mi. Nunc sodales tellus quis magna aliquet vel adipiscing augue semper. Nunc facilisis pharetra elit in ornare. Sed quis nunc sagittis turpis fringilla laoreet at a magna.
\begin{figure}
    \center \includegraphics[height=100pt]{bpositivo.png}
    \caption{Esta figura no muestra nada}
\end{figure}


\section*{Referencias}
\begin{thebibliography}{99}
    \bibitem{Libro_Visual} Francisco Javier Ceballos: Enciclopedia de Microsoft Visual
        Basic. Editorial Ra-Ma. Madrid, 1999.
    \bibitem{PHPBuch} Dieter Staas: PHP 5 Espresso!. Franzis Verlag GmbH.
        Poing, 2004.
    \bibitem{Ralph} Ralph Pfeiffer. Diplomarbeit: Planung und
        Erstellung einer im Höhenwinkel nachführbaren Photovoltaikanlage
        sowie Realisierung der Steuerung und Messwerterfassung eines
        Blockheizkraftwerkes in einem bestehenden regenerativen
        Energienetzwerk. Fachhochschule Braunschweig/Wolfenbüttel,
        Fachbereich Versorgungstechnik, 27.04.2004
    \bibitem{Lars}Lars Ortlieb, Diplomarbeit: Vernetzung alternativer
        Energiesysteme unter Verwendung moderner Kommunikationstechnik,
        Fachhochschule Braunschweig/Wolfenbüttel, Fachbereich
        Versorgungstechnik, 03.09.2003
    \bibitem{Schuamcher} Mike Schumacher. Datentechnische Erfassung einer
        nachgeführten Photovoltaikanlage in einem alternativen
        Energieverbund mittels moderner Kommunikationstechnologie.
        Fachhochschule Braunschweig/Wolfenbüttel, Fachbereich
        Versongungstechnik, 29.04.2004
    \bibitem{TAC} TAC Xenta 511 Engineering Manual
    \bibitem{TACWeb} TAC Web-Seite: \url{http://www.tac-global.com}
    \bibitem{PHP} PHP Web-Seite: \url{http://www.php.net}
\end{thebibliography}

\section*{Código Fuente}
    \label{codigo-compresionBruta2}
    \lstinputlisting[caption=compresionBruta2.m,mathescape=false]{../src/compresionBruta2.m}
\end{document}
