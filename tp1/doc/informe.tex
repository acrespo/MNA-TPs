\documentclass[twocolumn,a4paper,10pt]{article}

\usepackage[utf8]{inputenc}
\usepackage{t1enc}
\usepackage[spanish]{babel}
\usepackage[pdftex,usenames,dvipsnames]{color}
\usepackage[pdftex]{graphicx}
\usepackage{enumerate}
\usepackage{url}
\usepackage{amsmath}
\usepackage{amsfonts}
\usepackage{amssymb}
\usepackage[table]{xcolor}
\usepackage[small,bf]{caption}
\usepackage{float}
\usepackage{subfig}
\usepackage{bm}
\usepackage{fancyhdr}
\usepackage{times}
\usepackage{titlesec}
\usepackage[numbers]{natbib}
\usepackage{titling}
\usepackage{listings}

\renewcommand{\lstlistingname}{Código Fuente}

%%% Listings
\lstloadlanguages{Octave} 
\lstdefinelanguage{MyOctave}[]{Octave}{% 
	deletekeywords={beta,det},
	morekeywords={repmat}
} 
\lstset{ %
	language=MyOctave,
	stringstyle=\ttfamily,
	showstringspaces = false,
	basicstyle=\footnotesize\ttfamily,
	commentstyle=\color{gray},
	keywordstyle=\bfseries,
	numbers=left,
	numberstyle=\ttfamily\footnotesize,
	stepnumber=1,                   % the step between two line-numbers. If it's 1 each line will be numbered
	framexleftmargin=0.20cm,
	numbersep=0.37cm,               % how far the line-numbers are from the code
	backgroundcolor=\color{white},
	showspaces=false,
	showtabs=false,
	frame=l,
	tabsize=4,
	captionpos=b,                   % sets the caption-position to bottom
	breaklines=true,                % sets automatic line breaking
	breakatwhitespace=false,        % sets if automatic breaks should only happen at whitespace
	mathescape=true
}

\def\customabstract{\vspace{.5em}
    {\small\center{\textbf{RESUMEN}} \\[0.5em] \relax%
}}
\def\endkeywords{\par}

\def\keywords{\vspace{.5em}
    {\textit{Palabras clave: } 
}}
\def\endkeywords{\par}

\titleformat{\section}{\small\center\bfseries}{\thesection.}{0.5em}{\normalsize\uppercase}
\titleformat{\subsection}{\small\center\bfseries}{}{0.5em}{\small\uppercase}
\renewcommand{\thesection}{\Roman{section}}
\renewcommand{\bibsection}{}

% TITLE Configuration
\setlength{\droptitle}{-30pt}
\pretitle{\begin{center}\Huge\begin{rmfamily}}
\posttitle{\par\end{rmfamily}\end{center}\vskip 0.5em}
\preauthor{\begin{center}
        \large \lineskip 0.5em%
\begin{tabular}[t]{c}}
\postauthor{\end{tabular}\normalsize 
    \\[1em] Estudiantes del Instituto Tecnológico de Buenos Aires
\par\end{center}}
\predate{\begin{center}\small}
\postdate{\par\end{center}}

% Headers
\addtolength{\voffset}{-40pt}
\addtolength{\textheight}{80pt}
\renewcommand{\headrulewidth}{0pt}
\fancyhead{}
\fancyfoot{}
\lhead{\small No publicado: Cátedra de Métodos Numéricos Avanzados (ITBA)}
\rhead{\small \thepage}
\cfoot{\small Copyright \copyright 2012 ITBA}

% Metadata
\title{Autovalores y Compresión de Imágenes}
\date{20 de Septiembre de 2012}
\author{Civile, Juan Pablo \and Crespo, Álvaro \and Ordano, Esteban }

\begin{document}

\pagestyle{fancy}
\maketitle
\thispagestyle{fancy}

\begin{customabstract}
\textbf{
    Analizamos una serie de tecnicas de compresion de imagenes en blanco y negro.
}
\end{customabstract}

\begin{keywords}
paper, paper, paper
\end{keywords}


\section{Introducci\'on}

\section{Metodología}
% Pueden ser varias secciones

\subsection{Compresi\'on bruta 1: la imagen media}
\label{sec:compresion1}

Una forma muy simple de comprimir una imagen es la siguiente. Dividir la imagen en bloques no solapados de, por ej., 16 x 16. En la imagen de Lena (de 512 x 512) 
hay 1024 bloques: dar la imagen es equivalente a dar 256 n\'umeros por cada bloque. Una forma burda de comprimir la imagen es dar lo mismos 256 n\'umeros para todos 
los bloques, comprimiendo la imagen (con p\'erdida obviamente) de 256 kbytes a 256 bytes. Una posibilidad es dar los valores correspondientes a un ``bloque promedio''.\\

Nuestro c\'odigo de Octave (c\'odigo fuente \ref{compresionBruta1}) permite realizar dicha operaci\'on. \\

La funci\'on $imread$ lee im\'agenes en matrices. Al tratarse en este caso de una imagen de 512 x 512 en escala de grises, el resultado es una matriz de 512 filas y 
512 columnas en el que cada elemento es un entero que puede ser de 8 o 16 bits. La funci\'on $im2col$ toma esa matriz y divide en bloques de alg\'un tamaño dado en 
columnas seg\'un el modo que se desee (en nuestro caso, nos interesa que los bloqueas no se solapen). El resultado es la matriz que denominamos $b$, de 256 filas y 
1024 columnas. Se puede que la matriz $b$ tiene en cada columna un vector asociado a un bloque.  \\

Luego se procede a calcular la media aritm\'etica entre los bloques, es decir, el ``bloque promedio''. Para esto, utilizamos la funci\'on $mean$, que aplicada a 
matrices devuelve un vector con las medias de cada columna. Pero dado que queremos la media entre los bloques (y no la media de cada bloque) debemos aplicarle
$mean$ a la traspuesta de la matriz $b$. Se puede pensar que de esta forma estamos obteniendo la media de cada componente del ``bloque promedio''. \\

Bien, ahora tenemos un vector con el ``bloque promedio'', $m$, y tenemos volver a formar una matriz de 512 filas y 512 columnas. Para ello utilizaremos la funci\'on
$col2im$ que funciona de manera opuesta a $im2col$. Esta funci\'on toma una matriz y reagrupa sus columnas en bloques de un tamaño dado, formando una matriz del 
tamaño especificado seg\'un un criterio, que en nuestro caso es que los bloques no se solapen. Pero para utilizar esta funci\'on necesitamos la matriz cuyas 
columnas representen los bloques de la matriz (e imagen) final. Lo que podemos hacer es trasponer el vector $m$, nuestro ``bloque promedio'', y replicarlo de forma
de lograr tener la cantidad de bloques necesarios, es decir 1024. Esto se logra de manera sencilla utilizando la funci\'on $repmat$, la cual forma una matriz de un
tamaño dado con copias de una determinada matriz. En nuestro caso, llamamos a $repmat$ con la traspuesta de $m$, y 1 x 1024 como tamaño de matriz, ya que tenemos un 
vector columna de 256 elementos y queremos replicarlo 1024 veces para formar una matriz de 256 filas y 1024 columnas.\\

Luego, podemos aplicar $col2im$ a la matriz $M$ que devuelta por la funci\'on $repmat$, lo que nos da como resulta una matriz $d$. Solo resta redondear y los n\'umeros 
de dicha matriz y convertirlos a enteros de 8 bits (para que queden en el rango 0 - 255). Para ello, basta aplicar la funci\'on $uint8$ a la matriz $d$.

\subsection{Compresi\'on bruta 2: PCA}
\label{sec:compresion2}

Desde el punto de vista del  \'algebra lineal, la imagen dividida en bloques de 16 x 16, como en el caso anterior, se puede considerar simplemente como
1024 vectores de 256 elementos cada uno (la matriz b del código). Es decir, tenemos vectores en $\mathbb{R}^{256}$ . Dada una base cualquiera de $\mathbb{R}^{256}$,
 que cuenta con 256 vectores linealmente independientes, podemos representar cada uno de los bloques de la imagen por su proyecci\'on sobre cada vector en la base. 
Es decir, si queremos darle la imagen a alguien, le podemos dar la base y los 256 n\'umeros correspondientes a la proyeccci\'on sobre cada vector de la misma. \\

Para reducir el problema, podríamos pensar en usar $N < 256$ vectores linealmente independientes en la base. De esta manera, para comunicar la imagen s\'olo
tenemos que dar N vectores en la base y N n\'umeros por cada bloque. El problema es que usando s\'olo N vectores no se puede representar todo $\mathbb{R}^{256}$ y la 
imagen comunicada no ser\'ia exactamente igual a la original. Se plantea entonces el interrogante de como elegir los N vectores de manera de minimizar la p\'erdida 
de calidad de la imagen. La respuesta la da lo que se denomina \textit{Principal Component Analysis} y, en nuestro caso, es m\'as o menos como sigue.\\

Introducimos el famoso concepto de la Estadística, la Covarianza. La covarianza nos da una idea de qu\'e tanto cambian juntas dos variables. Si tenemos, por ej., 
dos variables aleatorias Y, Z la covarianza se define como,

\[ cov(Y, Z) = E [(Y - E(Y )) (Z - E(Z))] \]

donde $E(x)$ es la esperanza matem\'atica o valor esperado. \\

En nuestro caso, por ej., podemos considerar a cada uno de los 256 n\'umeros que representan el nivel de gris de un pixel en un bloque como un variable aleatoria.
Llamemos $X_{i}, i = 1, 2, \dotsc, 256$, a dichas variables aleatorias. Luego, podemos definir una matriz $C = (C_{ij} )$ de la siguiente manera,

\[C_{ij} = Cov(X_{i}, X_{j})  \]

A $\textbf{C}$ se la denomina matriz de covarianza.\\

Sin embargo, nosotros no tenemos distribuciones de probabilidad que nos permitan calcular la matriz covarianza. S\'olo tenemos, en el ejemplo de la imagen
de Lena, 1024 muestras de cada variable aleatoria: una por cada bloque. Esto nos permite calcular la matriz de covarianza muestral $\widehat{C} = (\widehat{C}_{ij})$:

\[ \widehat{C}_{ij} =  \frac{1}{1024 - 1}\sum_{k=1}^{1024} (x_{ik} - \widehat{x}_{i}) (x_{jk} - \widehat{x}_{j}) \]

donde
\begin{itemize}
    \item $x_{ik}$ es la k-\'esima muestra de la i- \'esima variable aleatoria o, en el contexto de nuestra imagen, el nivel de gris del i-\'esimo pixel en el 
    k- \'esimo bloque,
    \item $\widehat{x}_{i}$ es la media muestral de la i- \'esima variable aleatoria o el nivel de gris del i-\'esimo pixel del “bloque promedio”, es decir,
    \[ \widehat{x}_{i} = \frac{1}{1024} \sum_{k=1}^{1024} x_{ik} \]
\end{itemize}

\paragraph*{}

La matriz de covarianza muestral $\widehat{C}_{ij}$ nos habla de qu\'e tanto var\'ian juntos los niveles de gris del i-\'esimo y el j-\'esimo pixel en los bloques 
de la imagen. Y ahora viene  el punto clave: vamos a calcular los autovalores y autovectores. Cada autovector nos da una direcci\'on caracter\'istica de cambio y 
el autovalor correspondiente  (siempre $\geq$ 0) 

\footnote{La matriz de covarianza tiene la particularidad de ser sim\'etrica y tener los elementos de la diagonal no negativos, por ser las 
varianzas de cada variable aleatoria. Bas\'andose en estos datos, se puede demostrar que la matriz de covarianza es semi-definida positiva. Y una matriz semi-definida 
positiva y sim\'etrica tiene todos sus autovalores mayores o iguales a cero.}

 nos dice qu\'e tan importante es el cambio en dicha direcci\'on. En el contexto de la imagen de Lena, cada autovector es una \textit{autoimagen}
que nos habla de un esquema de cambio caracter\'istico de los niveles de gris de los bloques y cada autovalor nos habla de la importancia de ese esquema frente 
a los dem\'as. \\

Vamos a unir todo esto con la compresi\'on de la imagen.Los autovectores conforman la base de $\mathbb{R}^{256}$ de la que habl\'abamos m\'as arriba. 
Los autovalores nos dicen c\'omo elegir un subconjunto de esos autovectores: elegimos aquellos que tienen el mayor autovalor. \\

Nuestro c\'odigo de Octave (c\'odigo fuente \ref{compresionBruta2} muestra c\'omo realizar la compresi\'on usando estas ideas. \\

El comienzo del programa es similar al m\'etodo de compresi\'on anterior. Seguimos exactamente los mismos pasos para obtener las matrices $b$, la matriz que tiene como 
columnas los 1024 bloques de 256 elementos que representan la imagen, y $M$, la matriz que tiene como columnas 1024 copias del ``bloque promedio''.\\ 

Podemos ver como se usa la función $cov$ para calcular la matrix de covarianza muestral $\widehat{C}$. A la que luego se le calculan los autovalores y autovectores
utilizando una funci\'on recibida como par\'ametro del programa, $eig\_function$, que simplemente puede ser la funci\'on $eig$ de Octave o una definida por el usuario.
Utilizando la función $diag$ se extraen en un vector los autovalores de la diagonal de la matriz $D$ retornada por $eig\_function$. Dicho vector es ordenado de 
mayor a menor (de forma \textit{descendente}) usando la funci\'on $sort$ y luego se vuelve a utilizar la funci\'on $diag$ para obtener una matriz diagonal, cuya 
diagonal principal sea el vector ordenado. Una vez ordenados los autovalores, resta ordenar de la misma forma los autovectores. Esto se consiguen haciendo $V = V(:,i)$,
donde $i$ es el vector de los \'indices originales que es una de las cosas que devuelve la funci\'on $sort$ al ordenar a los autovalores.\\

Habiendo hecho esto, se procede a obtener las N proyecciones de los N autovectores correspondientes a los N autovalores más grandes. Esto se logra a través de la 
l\'inea $pv(:,i) = V(:,i).' * ds$. \\

Luego, se le suman estas proyecciones a los valores de la matriz $M$ del caso anterior (y llam\'andola ahora $d$), la cual contiene el ``gris promedio'' para todos 
los vectores. Esto se logra con un bloque $for$ de 1 a 1024, la cantidad de bloques en que dividimos la imagen, que se corresponde con la cantidad de columnas 
de la matriz $d$. En cada iteración de dicho bloque $for$, se deben sumar las N proyeccciones deseadas. Para ello, se necesita otro bloque $for$ de 1 a N, la cantidad 
de proyecciones. La suma de las proyecciones se consigue f\'acilmente mediante $d(:,k)\: += pv(k,i) * V(:,i)$. \\

Finalmente, s\'olo queda volver a juntar las columnas de la matriz $d$ en bloques no solapados para volver a formar la imagen. Esto se logra usando la funci\'on 
$col2im$ como lo hicimos anteriormente para la primera compresi\'on. Al igual que antes, se redondean y convierten a enteros de 8 bits los elementos de la matriz
resultante, usando $uit8$.\\

\section{Resultados}
% Podria ser Resultados y Conclusiones

\newcommand{\lena}[2]{
    \begin{figure}[H]
        \label{lena#1}
        \includegraphics[width=\linewidth]{images/lena#1.png}
        \caption{#2}
    \end{figure}
}

\lena{512}{La imagen original de Lena}

\lena{-bruta}{El resultado de aplicar la primer compresion (\ref{sec:compresion1})}

\lena{-eig-1}{El resultado de aplicar la segunda compresion (\ref{sec:compresion2}) con 1 autovector}
\lena{-eig-2}{El resultado de aplicar la segunda compresion (\ref{sec:compresion2}) con 2 autovectores}
\lena{-eig-3}{El resultado de aplicar la segunda compresion (\ref{sec:compresion2}) con 3 autovectores}
\lena{-eig-4}{El resultado de aplicar la segunda compresion (\ref{sec:compresion2}) con 4 autovectores}
\lena{-eig-16}{El resultado de aplicar la segunda compresion (\ref{sec:compresion2}) con 16 autovectores}
\lena{-eig-32}{El resultado de aplicar la segunda compresion (\ref{sec:compresion2}) con 32 autovectores}
\lena{-eig-128}{El resultado de aplicar la segunda compresion (\ref{sec:compresion2}) con 128 autovectores}

\lena{-qr-1}{El resultado de aplicar la segunda compresion (\ref{sec:compresion2}) con un solo autovector y nuestra implementacion del metodo QR}


\section{Conclusiones}
% Podria ser Resultados y Conclusiones

\section*{Referencias}
\begin{thebibliography}{99}
    \bibitem{Libro_Visual} Francisco Javier Ceballos: Enciclopedia de Microsoft Visual
        Basic. Editorial Ra-Ma. Madrid, 1999.
    \bibitem{PHPBuch} Dieter Staas: PHP 5 Espresso!. Franzis Verlag GmbH.
        Poing, 2004.
    \bibitem{Ralph} Ralph Pfeiffer. Diplomarbeit: Planung und
        Erstellung einer im Höhenwinkel nachführbaren Photovoltaikanlage
        sowie Realisierung der Steuerung und Messwerterfassung eines
        Blockheizkraftwerkes in einem bestehenden regenerativen
        Energienetzwerk. Fachhochschule Braunschweig/Wolfenbüttel,
        Fachbereich Versorgungstechnik, 27.04.2004
    \bibitem{Lars}Lars Ortlieb, Diplomarbeit: Vernetzung alternativer
        Energiesysteme unter Verwendung moderner Kommunikationstechnik,
        Fachhochschule Braunschweig/Wolfenbüttel, Fachbereich
        Versorgungstechnik, 03.09.2003
    \bibitem{Schuamcher} Mike Schumacher. Datentechnische Erfassung einer
        nachgeführten Photovoltaikanlage in einem alternativen
        Energieverbund mittels moderner Kommunikationstechnologie.
        Fachhochschule Braunschweig/Wolfenbüttel, Fachbereich
        Versongungstechnik, 29.04.2004
    \bibitem{TAC} TAC Xenta 511 Engineering Manual
    \bibitem{TACWeb} TAC Web-Seite: \url{http://www.tac-global.com}
    \bibitem{PHP} PHP Web-Seite: \url{http://www.php.net}
\end{thebibliography}

\section*{Código Fuente}
    \lstinputlisting[caption=compresionBruta1.m,label=compresionBruta1,mathescape=false]{../src/compresionBruta1.m}
    \lstinputlisting[caption=compresionBruta2.m,label=compresionBruta2,mathescape=false]{../src/compresionBruta2.m}
\end{document}
